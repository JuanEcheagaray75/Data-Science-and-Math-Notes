\documentclass{article}
\usepackage[utf8]{inputenc}
\usepackage[spanish]{babel}
\usepackage[]{amsthm}
\usepackage{amsmath}
\usepackage[]{amssymb}
\usepackage{graphicx}
\usepackage{wrapfig}
\usepackage[letterpaper, margin=1.5in]{geometry}
\usepackage[hidelinks]{hyperref}
\decimalpoint

\begin{document}
    \begin{titlepage}
        \begin{center}
            \begin{figure}
                \centering
                \includegraphics[scale=0.13]{img/logo_itesm.png}\\ % Logo de la institución
            \end{figure}
        \vspace{5cm}
        \LARGE{Instituto Tecnológico y de Estudios Superiores de Monterrey}\\
        \fontsize{12}{14}\selectfont
        \vspace{1cm}
        \textbf{Actividad: Distribución de Poisson}\\ % Nombre de la tarea
        \vspace{0.7cm}
        \begin{table}[h!]
            \centering
            \begin{tabular}{ ||c|c|| }
                \hline
                Nombre & Matrícula \\
                \hline
                Juan Pablo Echeagaray González & A00830646 \\
                \hline
                Verónica Victoria García De la Fuente & A00830383 \\
                \hline
                Emily Rebeca Méndez Cruz & A00830768 \\
                \hline
                Eugenio Santiesteban Zolezzi & A01720932 \\
                \hline
                Daniel de Zamacona Madero & A01570576 \\
                \hline
            \end{tabular}
        \end{table}
        \vspace{0.7cm}
        Optimización Estocástica\\ % Materia
        \vspace{0.2cm}
        MA2004B\\ % Clave de la materia
        \vspace{0.2cm}
        Dr. Fernando Elizalde Ramírez\\ % Nombre del profesor
        \vspace{0.7cm}
        21 de agosto del 2022\\ % Fecha de entrega
        \end{center}
    \end{titlepage}

    \section{Problemas}

        \subsection{Problema 1}

            Suponga que $X$ tiene una distribución de Poisson con media 4. Calcule las posibilidades siguientes:
            \begin{enumerate}
                \item $P(X = 0)$
                \item $P(X \leq 2)$
                \item $P(X = 4)$
                \item $P(X = 8)$
            \end{enumerate}

            \begin{gather*}
                X \sim \text{Poisson}(\mu = 4) \\
                P(X = 0) = \exp(-4) \frac{4^{0}}{0!} = \exp(-4) \approx 0.01831 \\
                P(X \leq 2) = \sum_{i=0}^{2} \exp(-i) \frac{4^{i}}{i!} \approx 0.2381 \\
                P(X = 4) = \exp{-4} \frac{4^{4}}{4!} = \frac{32}{3 \exp(4)} \approx 0.19536 \\
                P(X = 8) = \exp{-4} \frac{4^{8}}{8!} = \frac{512}{315 \exp(4)} \approx 0.02977
            \end{gather*}

        \subsection*{Problema 2}
            
            Se supone que el número de defectos en los rollos de tela de cierta industria textil es una variable aleatoria de Poisson con una media de 0.1 defectos por metro cuadrado.

            \begin{enumerate}
                \item ¿cuál es la probabilidad de tener dos defectos en un metro cuadrado de tela?
                \item ¿cuál es la probabilidad de tener un defecto en 10 metros cuadrados de tela?
                \item ¿cuál es la probabilidad de no tener defectos en 20 metros cuadrados de tela?
                \item ¿cuál es la probabilidad de tener al menos dos defectos en un metro cuadrado de tela?
            \end{enumerate}

            \begin{gather*}
                X \sim \text{Poisson}(\mu t = 0.1 t) \\
                P(X = 2) = \exp(-0.1) \frac{(0.1)^{2}}{2!} \approx 0.004524 \\
                P(X = 10) = \exp(-0.1 \cdot 10) \frac{(0.1 \cdot 10)^{1}}{1!} \approx 0.36787 \\
                P(X = 0) = \exp(-0.1 \cdot 20) \frac{(0.1 \cdot 20)^{0}}{0!} \approx 0.135335 \\
                P(X >= 2) = 1 - \sum_{i=0}^{0} \exp(0.1) \frac{(0.1)^{i}}{i!} \approx 0.000154653
            \end{gather*}


\end{document}