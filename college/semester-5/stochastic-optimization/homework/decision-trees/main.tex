\documentclass{article}
\usepackage[utf8]{inputenc}
\usepackage[spanish]{babel}
\usepackage[]{amsthm}
\usepackage{amsmath}
\usepackage[]{amssymb}
\usepackage{graphicx}
\usepackage{wrapfig}
\usepackage[letterpaper, margin=1.5in]{geometry}
\usepackage[hidelinks]{hyperref}

\usepackage{tikz}

\decimalpoint

\begin{document}
    \begin{titlepage}
        \begin{center}
            \begin{figure}
                \centering
                \includegraphics[scale=0.13]{logo_itesm.png}\\ % Logo de la institución
            \end{figure}
        \vspace{5cm}
        \LARGE{Instituto Tecnológico y de Estudios Superiores de Monterrey}\\
        \fontsize{12}{14}\selectfont
        \vspace{1cm}
        \textbf{MA2004. Árboles de decisión}\\ % Nombre de la tarea
        \vspace{0.7cm}
        \begin{table}[h!]
            \centering
            \begin{tabular}{ ||c|c|| }
                \hline
                Nombre & Matrícula \\
                \hline
                Juan Pablo Echeagaray González & A00830646 \\
                \hline
                Verónica Victoria García De la Fuente & A00830383 \\
                \hline
                Emily Rebeca Méndez Cruz & A00830768 \\
                \hline
                Eugenio Santiesteban Zolezzi & A01720932 \\
                \hline
                Daniel de Zamacona Madero & A01570576 \\
                \hline
            \end{tabular}
        \end{table}
        \vspace{0.7cm}
        Optimización estocástica\\ % Materia
        \vspace{0.2cm}
        MA2004B\\ % Clave de la materia
        \vspace{0.2cm}
        Fernando Elizalde Ramírez\\ % Nombre del profesor
        \vspace{0.7cm}
        12 de agosto del 2022\\ % Fecha de entrega
        \end{center}
    \end{titlepage}

    \section{Problema 1}

        Una empresa compra la materia prima a dos proveedores A y B, cuya calidad se muestra en la tabla siguiente:

        La probabilidad de recibir un lote del proveedor A en el que haya un 1\% de piezas defectuosas es del 80\%. Los pedidos que realiza la empresa ascienden a 1,000 piezas. Una pieza defectuosa puede ser reparada por 1 euro. Si bien tal y como indica la tabla la calidad del proveedor B es menor, éste está dispuesto a vender las 1,000 piezas por 10 euros menos que el proveedor A. Indique el proveedor que debe utilizar.

        \subsection{Procedimiento}

            \begin{table}[h!]
                \centering
                \begin{tabular}{ ||c|c|| }
                    \hline
                    Alternativas & Estados de naturaleza \\
                    \hline
                    Proveedor A & \shortstack{1\% defectuosas \\ 2\% defectuosas \\ 3\% defectuosas} \\
                    \hline
                    Proveedor B & \shortstack{1\% defectuosas \\ 2\% defectuosas \\ 3\% defectuosas} \\
                    \hline
                \end{tabular}
            \end{table}

            Ya que se han establecido las alternativas y los estados de naturaleza, podemos apoyarnos de un árbol de decisión para visualizar los posibles resultados con sus costos asociados, esto lo vemos en la figura \ref{fig:arbol1}

            \tikzset{
                treenode/.style = {shape=rectangle, draw, align=center, top color=white, bottom color=white},
                root/.style     = {treenode, bottom color=white, sibling distance=1cm},
                level 1/.style = {sibling distance=3cm},
                level 2/.style = {sibling distance=1.3cm},
                dummy/.style    = {circle,draw, sibling distance = 1mm},
                final/.style    = {sibling distance = 10cm},
            }

            \begin{figure}[!ht]
                \centering
                \begin{tikzpicture}[
                    grow = right,
                    edge from parent/.style = {draw, -latex},
                    level distance = 3cm,
                    sloped]
                    \node [root] {\$19}
                        child{
                            node [dummy] {\$19}
                                child{
                                    node [final] {\$30}
                                edge from parent node [above] {$0.3$}}
                                child{
                                    node [final] {\$20}
                                edge from parent node [above] {$0.3$}}
                                child{
                                    node [final] {\$10}
                                edge from parent node [above] {$0.4$}}
                            edge from parent node [below] {$P_B$}}
                        child{ 
                            node [dummy] {\$23}
                                child{
                                    node [final] {\$40}
                                edge from parent node [above] {$0.1$}}
                                child{
                                    node [final] {\$30}
                                edge from parent node [above] {$0.1$}}
                                child{
                                    node [final] {\$20}
                                edge from parent node [above] {$0.8$}}
                        edge from parent node [above] {$P_A$}};
                \end{tikzpicture}
                \caption{Árbol de decisión de los proveedores}
                \label{fig:arbol1}
            \end{figure}

    \section{Problema 2}

        El gerente de una empresa tiene dos diseños posibles para su nueva línea de cerebros electrónicos, la primera opción tiene un 80\% de probabilidades de producir el 70\% de cerebros electrónicos buenos y un 20\% de probabilidades de producir el 50\% de cerebros electrónicos buenos, siendo el coste de este diseño de 450,000 de euros. La segunda opción tiene una probabilidad del 70\% de producir el 70\% de cerebros electrónicos buenos y una probabilidad del 30\% de producir el 50\% de cerebros electrónicos buenos, el coste de este diseño asciende a 600,000 euros. El coste de cada cerebro electrónico es de 100 euros, si es bueno se vende por 250 euros, mientras que si es malo no tiene ningún valor.
        
        Conociendo que la previsión es de fabricar 50,000 cerebros electrónicos, decida el diseño que debe elegir el gerente de la empresa.
            
        \subsection{Procedimiento}

        \tikzset{
            treenode/.style = {shape=rectangle, draw, align=center, top color=white, bottom color=white},
            root/.style     = {treenode, bottom color=white, sibling distance=1cm},
            level 1/.style = {sibling distance=3cm},
            level 2/.style = {sibling distance=2cm},
            dummy/.style    = {circle,draw, sibling distance = 1mm},
            final/.style    = {sibling distance = 10cm},
        }

        \begin{figure}[!ht]
            \centering
            \begin{tikzpicture}[
                grow = right,
                edge from parent/.style = {draw, -latex},
                level distance = 5cm,
                sloped]
                \node [root] {2,400,000}
                    child{
                        node [dummy] {2,400,000}
                        child{
                            node [final] {650,000}
                            edge from parent node [below]{$P(50\% B) = 0.3$}
                            }
                        child{
                            node [final] {3,150,000}
                            edge from parent node [above]{$P(70\% B) = 0.7$}
                            }
                        edge from parent node [below]{Diseño 2}
                        }
                    child{
                        node [dummy] {2,600,000}
                        child{
                            node [final] {800,000}
                            edge from parent node [below]{$P(50\% B) = 0.2$}
                        }
                        child{
                            node [final] {3,300,000}
                            edge from parent node [above]{$P(70\% B) = 0.8$}
                        }
                        edge from parent node [above]{Diseño 1}
                    }
                ;
            \end{tikzpicture}
            \caption{Árbol de decisión de los diseños}
            \label{fig:arbol2}
        \end{figure}


    
\end{document}