\documentclass{article}
\usepackage[utf8]{inputenc}
\usepackage[spanish]{babel}
\usepackage[]{amsthm}
\usepackage{amsmath}
\usepackage[]{amssymb}
\usepackage{graphicx}
\usepackage{wrapfig}
\usepackage[letterpaper, margin=1.5in]{geometry}
\usepackage[hidelinks]{hyperref}
\decimalpoint

\begin{document}
    \begin{titlepage}
        \begin{center}
            \begin{figure}
                \centering
                \includegraphics[scale=0.13]{logo_itesm.png}\\ % Logo de la institución
            \end{figure}
        \vspace{5cm}
        \LARGE{Instituto Tecnológico y de Estudios Superiores de Monterrey}\\
        \fontsize{12}{14}\selectfont
        \vspace{1cm}
        \textbf{MA2004. Valor Esperado}\\ % Nombre de la tarea
        \vspace{0.7cm}
        \begin{table}[h!]
            \centering
            \begin{tabular}{ ||c|c|| }
                \hline
                Nombre & Matrícula \\
                \hline
                Juan Pablo Echeagaray González & A00830646 \\
                \hline
                Verónica Victoria García De la Fuente & A00830383 \\
                \hline
                Emily Rebeca Méndez Cruz & A00830768 \\
                \hline
                Eugenio Santiesteban Zolezzi & A01720932 \\
                \hline
                Daniel de Zamacona Madero & A01570576 \\
                \hline
            \end{tabular}
        \end{table}
        \vspace{0.7cm}
        Optimización estocástica\\ % Materia
        \vspace{0.2cm}
        MA2004B\\ % Clave de la materia
        \vspace{0.2cm}
        Fernando Elizalde Ramírez\\ % Nombre del profesor
        \vspace{0.7cm}
        12 de agosto del 2022\\ % Fecha de entrega
        \end{center}
    \end{titlepage}

    \section{Problema 1}

        María Rojas está considerando la posibilidad de abrir una pequeña tienda de vestidos en Fairbanks Avenue,  a pocas cuadras de la universidad. Ha localizado un buen centro comercial que atrae a estudiantes. Sus opciones son abrir una tienda pequeña, una tienda mediana o no abrirla en absoluto. El mercado para una tienda de vestidos puede ser bueno, regular o malo. Las probabilidades de estas tres posibilidades son 0.2 para un mercado bueno, 0.5 para un mercado regular y 0.3 para un mercado malo. La ganancia o pérdida neta para las tiendas mediana y pequeña en las diferentes condiciones del mercado se dan en la siguiente tabla. No abrir una tienda no tiene pérdida ni ganancia.

        \begin{figure}[!htbp]
            \centering
            \begin{tabular}{ |cccc| }
                \hline
                Alternativa & Mercado Bueno & Mercado Regular & Mercado Malo \\
                \hline
                Tienda Pequeña & $75,000$ & $25,000$ & $-40,000$ \\
                \hline
                Tienda Mediana & $100,000$ & $35,000$ & $-60,000$ \\
                \hline
                Tienda Grande & $0$ & $0$ & $0$ \\
                \hline
            \end{tabular}
            \caption{Tabla de alternativas: Problema 1}
            \label{tabla:problema1}
        \end{figure}

        \begin{enumerate}
            \item ¿Qué recomienda a María?
            \item Calcule el VEIP.
            \item Desarrolle la tabla de pérdida de oportunidad para esta situación. ¿Qué decisión se tomarán usando  el criterio de POE mínima?
        \end{enumerate}

        \subsection{Procedimiento}

            Calculamos primero los valores esperados asociados con las alternativas:

            \begin{gather*}
                VE_p = 75,000 \cdot 0.2 + 25,000 \cdot 0.5 + -40,000 \cdot 0.3 = \$ 15,500 \\
                VE_m = 100,000 \cdot 0.2 + 35,000 \cdot 0.5 + (-60,000) \cdot 0.3 = \$ 19,500 \\
                VE_g = 0 \cdot 0.2 + 0 \cdot 0.5 + 0 \cdot 0.3 = \$ 0 \\
                \max_i \{VE_i\} = VE_m = \$ 19,500
            \end{gather*}

            Ahora obtenemos el valor esperado con información perfecta:

            \begin{gather*}
                VECIP = 100,000 \cdot 0.2 + 35,000 \cdot 0.5 + 0 \cdot 0.3 = \$ 37,500 \\
                VEIP = VECIP - \max_i \{VE_i\} \\
                VEIP = \$ 37,500 - \$ 19,500 = \$ 18,000
            \end{gather*}

            Y finalmente calculamos una tabla de pérdidas de oportunidad esperadas:

            \begin{figure}[htbp!]
                \centering
                \begin{tabular}{ |cccc| }
                    \hline
                    Tienda / Mercado & $M_b$ & $M_r$ & $M_m$ \\
                    \hline
                    Tienda Pequeña & 25,000 & 10,000 & 40,000 \\
                    \hline
                    Tienda Mediana & 0 & 0 & 60000 \\
                    \hline
                    No tienda & 100,000 & 35,000 & 0 \\
                    \hline
                    $p$ & 0.2 & 0.5 & 0.3 \\
                    \hline
                \end{tabular}
                \caption{Tabla de pérdidas de oportunidad esperadas: Problema 1}
                \label{tabla:poe1}
            \end{figure}

            Con la tabla anterior ahora calculamos las pérdidas de oportunidad esperadas para cada una de las alternativas.

            \begin{gather*}
                POE_p = 0.2 \cdot (25,000) + 0.5 \cdot (10,000) + 0.3 \cdot (40,000) = \$ 22,000 \\
                POE_m = 0.2 \cdot (0) + 0.5 \cdot (0) + 0.3 \cdot (60,000) = \$18,000 \\
                POE_n = 0.2 \cdot (100,000) + 0.5 \cdot (35,000) + 0.3 \cdot (0) = \$37,500 \\
                \min_i \{POE_i\} = POE_m = \$18,000
            \end{gather*}

        Ya que tenemos toda esta información, podemos recomendarle un plan de acción a María. Bajo el criterio del valor esperado, la mejor opción es La tienda mediana es la que tiene el valor esperado más alto de \$19,500. En caso de que quisiéramos más información, podríamos realizar un estudio, pero este no podría superar un costo mayor de \$18,000.

        Si en cambio usáramos el criterio de pérdida de oportunidad de espera mínima, la mejor opción seguirá siendo la tienda mediana, con esta podríamos esperar perder solamente \$18,000.

    \section{Problema 2}

        Se le presenta la oportunidad de invertir en tres fondos mutuos: de servicios, de crecimiento agresivo, y global. El valor de su inversión cambiará según las condiciones del mercado. Hay 10\% de probabilidades de que el mercado baje; 50\% de que permanezca moderado, y 40\% de que funcione bien. La siguiente tabla proporciona el cambio porcentual del valor de la inversión en las tres condiciones:

        \begin{figure}[!htbp]
            \centering
            \begin{tabular}{ |cccc| }
                \hline
                Alternativa & Mercado bajista (\%) & Mercado Regular (\%) & Mercado Malo (\%) \\
                \hline
                Servicios & 5 & 7 & 8 \\
                \hline
                Crecimiento agresivo & -10 & 5 & 30 \\
                \hline
                Global & 2 & 7 & 20 \\
                \hline
            \end{tabular}
            \caption{Tabla de alternativas: Problema 2}
            \label{tabla:problema2}
        \end{figure}

        \begin{enumerate} 
            \item ¿Cuál fondo mutuo debe seleccionar?
            \item Calcule el VEIP.
            \item Desarrolle la tabla de pérdida de oportunidad para esta situación. ¿Qué decisión se tomarán usando  el criterio de POE mínima?
        \end{enumerate}

        \subsection{Procedimiento}

            Calculamos primero los valores esperados asociados con las alternativas:
            \begin{gather*}
                VE_s = 0.1 \cdot (5) + 0.5 \cdot (7) + 0.4 \cdot (8) = 7.2\% \\
                VE_a = 0.1 \cdot (-10) + 0.5 \cdot (5) + 0.4 \cdot (30) = 13.5\% \\
                VE_g = 0.1 \cdot (2) + 0.5 \cdot (7) + 0.4 \cdot (20) = 11.7\% \\
                \max_i \{VE_i\} = VE_a = 13.5\%
            \end{gather*}

            \begin{gather*}
                VECIP = 0.1 \cdot (0.5) + 0.5 \cdot (7) + 0.4 \cdot (30) = 16\% \\
                VEIP = VECIP - \max_i \{VE_i\} \\
                VEIP = 16\% - 13.5\% = 2.5\%
            \end{gather*}

            \begin{figure}[htbp!]
                \centering
                \begin{tabular}{ |cccc| }
                    \hline
                    Alternativa / Mercado & $M_b$ & $M_m$ & $M_a$ \\
                    \hline
                    S & 0 & 0 & 22 \\
                    \hline
                    A & 15 & 2 & 0 \\
                    \hline
                    G & 3 & 0 & 10 \\
                    \hline
                    $p$ & 0.1 & 0.5 & 0.4 \\
                    \hline
                \end{tabular}
                \caption{Tabla de pérdidas de oportunidad esperadas: Problema 1}
                \label{tabla:poe2}
            \end{figure}

            \begin{gather*}
                POE_s = 0.1 \cdot (0) + 0.5 \cdot (0) + 0.4 \cdot (22) = 8.8 \% \\
                POE_a = 0.1 \cdot (15) + 0.2 \cdot (0.2) + 0.4 \cdot (0) = 2.5 \% \\
                POE_g = 0.1 \cdot (3) + 0.5 \cdot (0) + 0.4 \cdot (10) = 4.3 \% \\
                \min_i \{POE_i\} = POE_a = 2.5 \%
            \end{gather*}
            
        El fondo mutuo que debemos de seleccionar es el de crecimiento agresivo con un valor esperado de crecimiento de 13.5\%, si quisiéramos un mayor nivel de confianza en nuestra estimación, podríamos realizar un estudio, pero este no puede representar un costo mayor del 2.5\% de nuestras inversiones; un criterio diferente que podemos seguir es el de la pérdida de oportunidad mínima, se llega a la misma decisión, la alternativa con la pérdida de oportunidad esperada mínima es la de la inversión en un crecimiento agresivo.

\end{document}